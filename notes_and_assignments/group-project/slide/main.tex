% !TEX TS-program = pdflatex
% !TEX encoding = UTF-8 Unicode
% 
% This file is a template using the "beamer" package to create slides for a talk or presentation
% - Talk at a conference/colloquium.
% - Talk length is about 20min.
% - Style is ornate.
% 
% MODIFIED by Jonathan Kew, 2008-07-06
% The header comments and encoding in this file were modified for inclusion with TeXworks.
% The content is otherwise unchanged from the original distributed with the beamer package.
\documentclass{beamer}
\input{header.tex}
\input{information.tex}
% 
% If you wish to uncover everything in a step-wise fashion, uncomment
% the following command: 
%\beamerdefaultoverlayspecification{<+->}
\begin{document}
% 
\begin{frame}
  \titlepage
\end{frame}
% \input{CV.tex}
% 
\begin{frame}
  \tableofcontents
  % You might wish to add the option [pausesections]
\end{frame}
% 
% 
% Structuring a talk is a difficult task and the following structure
% may not be suitable. Here are some rules that apply for this
% solution: 
% 
% - Exactly two or three sections (other than the summary).
% - At *most* three subsections per section.
% - Talk about 30s to 2min per frame. So there should be between about
%   15 and 30 frames, all told.
% 
% - A conference audience is likely to know very little of what you
%   are going to talk about. So *simplify*!
% - In a 20min talk, getting the main ideas across is hard
%   enough. Leave out details, even if it means being less precise than
%   you think necessary.
% - If you omit details that are vital to the proof/implementation,
%   just say so once. Everybody will be happy with that.
% 
% 
\section{Introduction}
\subsection{Background}
\begin{frame}
  \frametitle{The Premier League}
  \begin{itemize}
    \item Premier League: Top tier of English Football League System.
    \item 20 teams play 38 home and away matches.
    \item Globally renowned and challenging to predict outcomes.
  \end{itemize}
  
  \textbf{Background}
  
  \begin{enumerate}
    \item Outcome predictions involve expert analysis.
    \item Factors include team performance, player form, and tactics.
    \item Growing data, e.g., player touches, team running stats, manager experience.
  \end{enumerate}
\end{frame}

% All of the following is optional and typically not needed. 
% 
% 
% 
% 
% ----------------------------------------------------------------------
\section{Mathematical Modeling}
% 
% 
% 
% 
\subsection{Method 1: Entropy Weight Method in Football Team Evaluation}
\begin{frame}
  \frametitle{Overview of Entropy Weight Method in Football}
  % 
  % 
  % 
  % 
  % 
  % 
  \begin{enumerate}
    \item Introduction
          \begin{itemize}
            \item The Entropy Weight Method is a powerful analytical technique used in football team evaluation. It goes beyond traditional methods by considering the inherent information entropy within various performance attributes.
          \end{itemize}
    \item Key Characteristics
          \begin{itemize}
            \item Entropy: Reflects the degree of uncertainty or randomness within a dataset.
            \item Weight Assignment: Assigns weights to attributes based on their information entropy.

          \end{itemize}
    \item Objective
          \begin{itemize}
            \item The method aims to provide a nuanced evaluation, giving higher importance to attributes that contribute more to understanding a team's performance.
          \end{itemize}
  \end{enumerate}


\end{frame}
% 
% 
% 
% 
\begin{frame}
  \frametitle{Key Steps in Entropy Method}
  \begin{enumerate}
    \item \textbf{Data Collection and Attribute Selection}
    \item \textbf{Entropy Calculation:}
          \begin{itemize}
            \item Utilize mathematical formulas to calculate the entropy of each selected attribute.
            \item Entropy = $- \sum (p_i \cdot \log_2(p_i))$, where $p_i$ is the probability of each attribute value.
          \end{itemize}

    \item \textbf{Weight Assignment:}
          \begin{itemize}
            \item Assign weights to attributes based on their calculated entropy.
            \item Attributes with higher entropy receive lower weights, and vice versa.
            \item The sum of weights equals 1 for normalization.
          \end{itemize}

    \item \textbf{Outcome:}
          \begin{itemize}
            \item The result is a set of weights that reflect the relative importance of each attribute in evaluating a football team's performance.
          \end{itemize}
  \end{enumerate}
\end{frame}
% 
% 
\begin{frame}
  \frametitle{Entropy Method in Our Model}
  \begin{figure}
    \includegraphics[width=\textwidth]{img/entropy_method_flowchart.png}
  \end{figure}
  % 
  % 
\end{frame}
% 
% 
% 
% 
% 
% 
% 
% 
% 
% 
% 
% 
% 
% 
\subsection{Method 2: Gradient Ascend to Predict Winning}
\begin{frame}
  \frametitle{Gradient Ascend for Prediction}
\end{frame}
% 
% 
% 
% 
% 
% 
% 
% 
\section{Result Analysis}
\begin{frame}
  \frametitle{Result Analysis}
\end{frame}
% 
% 
% 
% 
% 
% 
% 
% 
% 
% 
% 
% 
% 
% 
% 
% 
\section{Future Work}
\begin{frame}
  \frametitle{Future Work}
\end{frame}
% 
% 
% 
% 
% 
% 
% 
% 
% 
% 
% 
% 
% 
% 
% 
% 
\end{document}


