\section{Model Assumptions}
% 
% 
% 
\subsection{Performance Assumptions (Correlation Analysis)}
\begin{itemize}
    \item  The number of wins positively correlates with the final league standing.
    \item  Teams with a higher goal difference (GF - GA) tend to achieve higher league positions.
    \item Drawn matches have a minimal impact on final league standings.
    \item Teams with a higher number of goals scored (GF) are more likely to finish in the top positions.
    \item  The defensive performance, measured by goals against (GA), influences the team's final standing.
    \item  The number of points earned directly correlates with the team's final position in the league.
\end{itemize}
\subsection{Consistency Hypothesis (Principal Component Analysis)}
% 
% 
% 
% 
\begin{itemize}
    \item Consistency in performance, as measured by a balanced distribution of wins, draws, and losses, is associated with a higher league position.
\end{itemize}
\paragraph{PCA can help identify patterns and relationships among these variables, which can contribute to understanding the consistency in team performance.}




\subsection{Historical Performance Hypothesis (Entropy Weighting)}
\begin{itemize}
    \item Teams with a consistent performance history over the years are likely to maintain their competitive positions.
\end{itemize}