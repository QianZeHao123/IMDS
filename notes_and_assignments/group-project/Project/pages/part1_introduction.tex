\section{Introduction}
% 
% 
\paragraph{The English Premier League is a ranking from one to twenty of the teams who won the most matches over the season. We are investigating the variables which factor into the ranking of the premier league; however, our dataset is fairly limited. We are examining historical data from previous seasons consisting of: the games won, lost and drawn by each team as well as the total number of goals for and against their team, and the goal difference. Ideally, we would analyse a dataset which featured variables not directly related to the ranking. In an attempt to enrich our dataset we performed the Fourier Transformation on the historical data from time domain to frequency domain. Time Domain feature extraction will allow us to examine the Time Domain evolution of the data. By analysing the time variability of metrics such as points, goals scored, goals conceded, etc., we hope to discover time-related patterns, such as seasonal changes or performance trends over a specific period. Of course, the most direct relationship is between the number of games won and the teams' placement on the league table, because this is how the table is curated. However, we did find some less obvious correlation; the teams' consistency in performance, as measured by a balanced distribution of wins, draws, and losses, is associated with a higher league position.  To test this hypothesis, we have made use of Principal Component Analysis (PCA), to find the most significant components, and the Entropy method (Entropy Weighting), with the intention of finding the overall weighting of the team's performance. With the goal of establishing that teams with a consistent performance history over the seasons is likely to allow them to maintain their competitive positions. PCA allows us to establish which were the most significant factors in achieving first place on the premier league table. This is appropriate for our dataset because our dataset's multicollinearity is very high, so we are using PCA to eliminate this issue. }