% !TEX TS-program = pdflatex
% !TEX encoding = UTF-8 Unicode
% 
% This is a simple template for a LaTeX document using the "article" class.
% See "book", "report", "letter" for other types of document.
% 
\documentclass[11pt]{article} % use larger type; default would be 10pt
% 
% 
% 
%%% The "real" document content comes below...
\input{header}
% 
\title{Introduction to Mathematics for Data Science \\ Personal Assignnment 1}
\author{Zehao Qian}
\begin{document}
\maketitle
% 
% 
% 
\section{Question 1}
\subsection{Modelling Bird Population Decline Due to Invasive Snakes}
% 
\paragraph{To model the bird population over time until extinction, I use a logistic growth model, which is commonly used to describe population growth and decline. The logistic growth model is expressed as:}
% 
\[P(t) = \frac{K}{1 + \frac{K - P_0}{P_0} \cdot e^{-rt}}\]
% 
\paragraph{Where:}
\begin{itemize}
    \item \(P(t)\) is the population at time \(t\).
    \item \(K\) is the carrying capacity, representing the maximum sustainable population size.
    \item \(P_0\) is the initial population at \(t = 0\).
    \item \(r\) is the growth rate parameter.
    \item \(t\) is time.
\end{itemize}
% 
% 
\paragraph{In your scenario, the bird population is declining due to the invasive snake species, so you'll need to use a negative growth rate (\(r < 0\)). The population starts at a certain level (\(P_0\)) and gradually approaches zero as time progresses.}

\paragraph{Here's a Python function that models the bird population over time using the logistic growth model and plots it:}
% 
% 
\begin{figure}[H]
    \centering
    \includegraphics[width=0.75\textwidth]{pic/BirdPopullationModel.png}
    \caption{Bird Prediction Graph}
\end{figure}
% 
% 
\end{document}
% 