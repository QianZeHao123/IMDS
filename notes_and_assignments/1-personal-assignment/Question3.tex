\section{Question 3}
\subsection{Gradient Estimation at $(0.5, 0.3)$ Using Linear Interpolation}
\paragraph{In this question I use the \textbf{finite difference method}. It is a commonly used numerical computational technique for estimating derivatives or partial derivatives of a function, especially when dealing with discrete data points.}
% 
% 
\paragraph{The finite difference method is typically applied in the following scenarios:}
% 
% 
\begin{itemize}
    \item A set of discrete data points but do not possess an analytical expression for the function.
    \item Compute the derivative of a function at a specific point, but lack an analytical expression for the derivative.
\end{itemize}
% 
\paragraph{I apply Central Difference Method (Here, h is a small step size):}
% 
% 
% 
\begin{equation}
    \nabla g(x,y)=(\frac{\partial g}{\partial x},\frac{\partial g}{\partial y})
\end{equation}
% 
% 
\begin{equation}
    \frac{\partial g}{\partial x}=\frac{g(x+h,y)-g(x-h,y)}{2 h}
\end{equation}
% 
% 
\begin{equation}
    \frac{\partial g}{\partial y}=\frac{g(x,y+h)-g(x,y-h)}{2 h}
\end{equation}
% 
% 
% 
% 
% 
% 
% 
% 
% 
% 
% 
% 
% 
% 
\subsection{Estimating Directional Derivative of g at $(0.5, 0.3)$ along the Vector $u=(1,-4)$}
% 
% 
% 
% 
\paragraph{To estimate the directional derivative of \(g\) at the point \((x, y) = (0.5, 0.3)\) along the vector \(\mathbf{u} = (1, -4)\), the directional derivative of \(g\) along the vector \(\mathbf{u} = (a, b)\) can be calculated using the gradient \(\nabla g\) as follows:}

\begin{equation}
    D_{\mathbf{u}} g = \nabla g \cdot \mathbf{u} = \frac{\partial g}{\partial x} \cdot a + \frac{\partial g}{\partial y} \cdot b
\end{equation}

\paragraph{\(\mathbf{u} = (1, -4)\) and I have estimated the gradient components as \(\frac{\partial g}{\partial x}\) and \(\frac{\partial g}{\partial y}\).}

You can now plug these values into the formula to estimate the directional derivative:
\begin{equation}
    D_{\mathbf{u}} g = \left(\frac{\partial g}{\partial x}\right) \cdot 1 + \left(\frac{\partial g}{\partial y}\right) \cdot (-4)
\end{equation}

\paragraph{Use the values for \(\frac{\partial g}{\partial x}\) and \(\frac{\partial g}{\partial y}\) to calculate \(D_{\mathbf{u}} g\).}
% 
% 
% 
% 
% 
% 
% 
% 
% 
% 
% 