\section{Question 2}
% 
% 
% 
\subsection{Derivative $f'(x)$ at $x = 7$}
% 
\paragraph{From the definition of the derivative, it can be seen that:}
% 
\begin{equation}
    f'(x)=\frac{f(x + \Delta x)-f(x)}{\Delta x}
\end{equation}
% 
% 
\paragraph{In the diagram I plotted at $x=7,7.2$, while $y=3.8,4.4$. So we let $\Delta x=0.2$}
% 
% 
% 
% 

% = \frac{f(x=7+\Delta x)-f(x=7)}{\Delta x} \\
\begin{align*}
    f'(x)
     & = \frac{f(7+\Delta x)-f(7)}{\Delta x} \\
     & = \frac{f(7.2)-f(7)}{0.2}             \\
     & = \frac{4.4-3.8}{0.2}                 \\
     & =3
\end{align*}

% 
% 
% 
% 
% 
\begin{figure}[H]
    \centering
    \includegraphics[width=0.7\textwidth]{pic/q2a.jpeg}
    \caption{$f(x)$'s value when $x=7,7.2$}
\end{figure}
% 
% 
% 
% 
% 
% 
% 
% 
% 
% 
% 
% 
% 
% 
% 
% 
% 
\subsection{The Estimated Value of $\int^6_2 f(x)dx$}
\begin{figure}[H]
    \centering
    \includegraphics[width=0.7\textwidth]{pic/q2b.jpeg}
    \caption{It is about 258 blocks from $x=2$ to $6$}
\end{figure}
% 
% 
% 
\paragraph{The area of each block is: $0.2 \times 0.2 = 0.04$, so the value of $ \int^6_2 f(x)dx $ is $ 253 \times 0.04 = \mathbf{10.12} $}
% 
% 
% 
% 
% 
% 
% 
% 
% 
\subsection{Draw the plot of function h}
% 



\tikzset{every picture/.style={line width=0.75pt}} %set default line width to 0.75pt        

\begin{tikzpicture}[x=0.75pt,y=0.75pt,yscale=-1,xscale=1]
    %uncomment if require: \path (0,300); %set diagram left start at 0, and has height of 300

    %Shape: Axis 2D [id:dp9944819171450985] 
    \draw  (118.09,141.78) -- (552,141.78)(275.7,8.4) -- (275.7,285.24) (545,136.78) -- (552,141.78) -- (545,146.78) (270.7,15.4) -- (275.7,8.4) -- (280.7,15.4)  ;
    %Curve Lines [id:da27355266451050975] 
    \draw    (187.56,227) .. controls (186.78,190.07) and (260.88,142.49) .. (275.7,141.78) .. controls (290.52,141.07) and (303.78,184.39) .. (348.24,184.39) .. controls (392.7,184.39) and (371.38,218.69) .. (403.62,218.48) .. controls (435.86,218.27) and (461.57,71.44) .. (467.81,60.79) ;
    %Straight Lines [id:da590418359643835] 
    \draw [color={rgb, 255:red, 208; green, 2; blue, 27 }  ,draw opacity=1 ] [dash pattern={on 0.84pt off 2.51pt}]  (392.31,184.74) -- (304.17,184.03) ;
    %Straight Lines [id:da6199838448049988] 
    \draw [color={rgb, 255:red, 208; green, 2; blue, 27 }  ,draw opacity=1 ] [dash pattern={on 0.84pt off 2.51pt}]  (447.69,218.83) -- (359.55,218.12) ;
    %Straight Lines [id:da4934879921966593] 
    \draw [color={rgb, 255:red, 208; green, 2; blue, 27 }  ,draw opacity=1 ] [dash pattern={on 0.84pt off 2.51pt}]  (319.77,142.13) -- (231.63,141.42) ;
    %Straight Lines [id:da04786413405641343] 
    \draw [color={rgb, 255:red, 208; green, 2; blue, 27 }  ,draw opacity=1 ] [dash pattern={on 0.84pt off 2.51pt}]  (348.24,184.39) -- (347.5,141.99) ;
    %Straight Lines [id:da1730175943856178] 
    \draw [color={rgb, 255:red, 208; green, 2; blue, 27 }  ,draw opacity=1 ] [dash pattern={on 0.84pt off 2.51pt}]  (403.62,218.48) -- (403.66,141.46) ;
    %Straight Lines [id:da6218480984588155] 
    \draw [color={rgb, 255:red, 208; green, 2; blue, 27 }  ,draw opacity=1 ] [dash pattern={on 0.84pt off 2.51pt}]  (203.59,190.78) -- (203.59,142.52) ;
    %Straight Lines [id:da684943730116574] 
    \draw [color={rgb, 255:red, 208; green, 2; blue, 27 }  ,draw opacity=1 ] [dash pattern={on 0.84pt off 2.51pt}]  (247.66,191.14) -- (159.52,190.43) ;

    % Text Node
    \draw (276.96,120.71) node [anchor=north west][inner sep=0.75pt]   [align=left] {O};
    % Text Node
    \draw (292.28,81.7) node [anchor=north west][inner sep=0.75pt]    {$h( x)$};
    % Text Node
    \draw (534.19,148.41) node [anchor=north west][inner sep=0.75pt]   [align=left] {X};
    % Text Node
    \draw (282.06,5.13) node [anchor=north west][inner sep=0.75pt]   [align=left] {Y};
    % Text Node
    \draw (325.97,118.98) node [anchor=north west][inner sep=0.75pt]    {$x=2$};
    % Text Node
    \draw (383.3,118.98) node [anchor=north west][inner sep=0.75pt]    {$x=3$};
    % Text Node
    \draw (171.8,119.52) node [anchor=north west][inner sep=0.75pt]    {$x=-2$};


\end{tikzpicture}
% 
% 
\paragraph{\textbf{Analytics:}}
\begin{itemize}
    \item For x < 0, the graph increases.
    \item At x = 0, there is a local maximum.
    \item For 0 < x < 2, the graph decreases.
    \item At x = 2, there is a local minimum.
    \item For 2 < x < 3, the graph decreases.
    \item At x = 3, there is an inflection point.
    \item For x > 3, the graph increases.
\end{itemize}
% 
% 
% 
% maximum
% minimum
% 
% b 258