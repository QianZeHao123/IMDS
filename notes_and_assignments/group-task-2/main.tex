% !TEX TS-program = pdflatex
% !TEX encoding = UTF-8 Unicode
% 
% This is a simple template for a LaTeX document using the "article" class.
% See "book", "report", "letter" for other types of document.
% 
\documentclass[11pt]{article} % use larger type; default would be 10pt
% 
% 
% 
%%% The "real" document content comes below...
\input{header}
% 
\title{Introduction to Math for DS Group Task 2}
\author{IMDS Group 24 \\ Zehao Qian, Mohammad Jamshaid Iqbal, Chloe Mendez}
\begin{document}
\maketitle
% 
% 
% \input{example}
\section{Question 1}
% 
% 
% 
\paragraph{If \( A = \begin{bmatrix}
        1 & 0 & 4 & 1  \\
        0 & 2 & 0 & 2  \\
        6 & 0 & 3 & 11 \\
    \end{bmatrix} \)
    ,
    \( B = \begin{bmatrix}
        7 & -1 & 2 \\
        1 & 1  & 0 \\
        2 & 0  & 1 \\
    \end{bmatrix} \). What is the product AB?}
% 
% 
% 
% 
% 
% 
% 
% 
% 
\subsection{AI Bot's Answer}
\begin{mdframed}[
        backgroundcolor=white,  % 背景颜色
        linecolor=black,        % 边框颜色
        leftmargin=5pt,         % 左边距
        rightmargin=5pt,        % 右边距
        linewidth=2pt           % 边框的宽度
    ]
    \paragraph{\textbf{ChatGPT}: To find the product \(AB\), you multiply matrix \(A\) by matrix \(B\) using the matrix multiplication rules. The element in the \(i\)-th row and \(j\)-th column of the product matrix \(AB\) is calculated by taking the dot product of the \(i\)-th row of matrix \(A\) and the \(j\)-th column of matrix \(B\).}
    % 
    \paragraph{Let's calculate the individual elements of the resulting matrix \(AB\):}
    % 
    \[ AB_{ij} = \sum_{k=1}^{n} A_{ik} \cdot B_{kj} \]
    % 
    \paragraph{Here, \(n\) is the number of columns in matrix \(A\) (which must be equal to the number of rows in matrix \(B\) for the multiplication to be defined).}
    \paragraph{\textcolor{red}{Until now, all of ChatGPT's responses are correct, and the matrix multiplication formula for \(AB\) is accurate, especially "the number of columns in matrix \(A\) (which must be equal to the number of rows in matrix \(B\) for the multiplication to be defined)".}}
    % 
    \paragraph{Given matrices \(A\) and \(B\):}
    % 
    \[ A = \begin{bmatrix}
            1 & 0 & 4 & 1  \\
            0 & 2 & 0 & 2  \\
            6 & 0 & 3 & 11 \\
        \end{bmatrix} \]

    \[ B = \begin{bmatrix}
            7 & -1 & 2 \\
            1 & 1  & 0 \\
            2 & 0  & 1 \\
        \end{bmatrix} \]
    % 
    \paragraph{Let's calculate the elements of \(AB\):}
    \paragraph{\textcolor{red}{Clearly, these two matrices cannot be multiplied.}}
    % 
    \[ AB_{11} = (1 \cdot 7) + (0 \cdot 1) + (4 \cdot 2) + (1 \cdot 2) = 7 + 0 + 8 + 2 = 17 \]
    % 
    \[ AB_{12} = (1 \cdot (-1)) + (0 \cdot 1) + (4 \cdot 0) + (1 \cdot 0) = -1 + 0 + 0 + 0 = -1 \]
    % 
    \[ AB_{13} = (1 \cdot 2) + (0 \cdot 0) + (4 \cdot 1) + (1 \cdot 1) = 2 + 0 + 4 + 1 = 7 \]
    % 
    \[ AB_{21} = (0 \cdot 7) + (2 \cdot 1) + (0 \cdot 2) + (2 \cdot 0) = 0 + 2 + 0 + 0 = 2 \]
    % 
    \[ AB_{22} = (0 \cdot (-1)) + (2 \cdot 1) + (0 \cdot 0) + (2 \cdot 0) = 0 + 2 + 0 + 0 = 2 \]
    % 
    \[ AB_{23} = (0 \cdot 2) + (2 \cdot 0) + (0 \cdot 1) + (2 \cdot 1) = 0 + 0 + 0 + 2 = 2 \]
    % 
    \[ AB_{31} = (6 \cdot 7) + (0 \cdot 1) + (3 \cdot 2) + (11 \cdot 0) = 42 + 0 + 6 + 0 = 48 \]
    % 
    \[ AB_{32} = (6 \cdot (-1)) + (0 \cdot 1) + (3 \cdot 0) + (11 \cdot 0) = -6 + 0 + 0 + 0 = -6 \]
    % 
    \[ AB_{33} = (6 \cdot 2) + (0 \cdot 0) + (3 \cdot 1) + (11 \cdot 1) = 12 + 0 + 3 + 11 = 26 \]
    % 
    \paragraph{Therefore, the product \(AB\) is:}
    \paragraph{\textcolor{red}{ChatGPT mistakenly treated matrix \(B\) as a \(4 \times 4\) matrix. Despite this, the calculations provided above are still incorrect, resulting in \(AB\) being a \(3 \times 3\) matrix.}}
    \[ AB = \begin{bmatrix}
            17 & -1 & 7  \\
            2  & 2  & 2  \\
            48 & -6 & 26 \\
        \end{bmatrix} \]

\end{mdframed}
% 
% 
% 
% 
% 
% 
% 
% 
\subsection{Analytics}
% 
% 
\paragraph{In addition to the analysis of the AI mentioned above, our team also wrote a program using numpy to compute matrix multiplication. However, when running the program, we encountered an error. The result is as follows:}
% 
% 
% 
% 
$$$$
% 
% 
\begin{lstlisting}[style=pystyle]
import numpy as np

# Define the matrices A and B
A = np.array([
    [1, 0, 4, 1],
    [0, 2, 0, 2],
    [6, 0, 3, 11]
])

B = np.array([
    [7, -1, 2],
    [1, 1, 0],
    [2, 0, 1]
])

# Calculate the product of A and B
AB_product = np.dot(A, B)
AB_product    
\end{lstlisting}
% 
% 
% 
% 
\paragraph{\textcolor{red}{Return ValueError: shapes (3,4) and (3,3) not aligned: 4 (dim 1) != 3 (dim 0)}}
% 
% 
\paragraph{\textbf{Analytics:} The product of two matrices AB is undefined if the number of columns in the first matrix A does not match the number of rows in the second matrix B. In this case, matrix A has 4 columns, while matrix B has 3 rows, so their product cannot be computed. Matrix multiplication requires that the number of columns in the first matrix be equal to the number of rows in the second matrix. If there's a third matrix C that should be involved to make the multiplication possible, please provide it, otherwise matrix A and B as given cannot be multiplied.}
% 
% 
% 
% 
% 
% 
% 
% 
% 
% 
% 
% 
% 
% 
\section{Question 2}
\paragraph{What is the dimension of the span of the vectors $(5,7,9,0)$, $(2,5,0,1)$, $(0,0,0,1)$ and $(7,12,9,3)$?}
% 
% 
% 
% 
% 
% 
\subsection{AI Bot's Answer}
\begin{mdframed}[
        backgroundcolor=white,  % 背景颜色
        linecolor=black,        % 边框颜色
        leftmargin=5pt,         % 左边距
        rightmargin=5pt,        % 右边距
        linewidth=2pt           % 边框的宽度
    ]
    \paragraph{\textbf{ChatGPT}: To find the dimension of the span of the given vectors, we can create a matrix using these vectors as columns and then find the rank of the matrix.}
    % 
    \paragraph{The given vectors are:}
    % 
    \[
        \begin{bmatrix}
            5 \\
            7 \\
            9 \\
            0
        \end{bmatrix},
        \begin{bmatrix}
            2 \\
            5 \\
            0 \\
            1
        \end{bmatrix},
        \begin{bmatrix}
            0 \\
            0 \\
            0 \\
            1
        \end{bmatrix},
        \begin{bmatrix}
            7  \\
            12 \\
            9  \\
            3
        \end{bmatrix}
    \]
    % 
    \paragraph{Let's create a matrix \(M\) using these vectors:}

    \[ M = \begin{bmatrix}
            5 & 2 & 0 & 7  \\
            7 & 5 & 0 & 12 \\
            9 & 0 & 0 & 9  \\
            0 & 1 & 1 & 3
        \end{bmatrix} \]
    % 
    \paragraph{Now, we will perform row operations to bring the matrix to its echelon form or reduced row-echelon form and find the rank.}
    % 
    \paragraph{\textcolor{red}{To determine how many linearly independent vectors are among these four vectors, arrange them into a matrix and perform row operations to transform it into row-echelon form (REF) or reduced row-echelon form (RREF). ChatGPT choose the correct answer.}}
    % 
    \[ RREF(M) = \begin{bmatrix}
            1 & 0 & 0 & 1 \\
            0 & 1 & 1 & 2 \\
            0 & 0 & 0 & 0 \\
            0 & 0 & 0 & 0
        \end{bmatrix} \]

    \paragraph{The rank of this matrix (and hence the vectors) is 2. Therefore, the dimension of the span of the given vectors is 2.}
    % 
    \paragraph{\textcolor{red}{However, ChatGPT made an error in calculating RREF(M) during the matrix transformation, consequently leading to an incorrect rank for the matrix.}}
    % 
    $$$$

\end{mdframed}
% 
% 
% 
% 
% 
% 
% 
% 
% 
% 
% 
% 
\subsection{Analytics}
% 
% 
% 
\paragraph{We used the "np.linalg.matrix\_rank" function in NumPy to calculate the rank of matrix \(M\).}
% 
$$$$
% 
% 
\begin{lstlisting}[style=pystyle]
import numpy as np

# Define the vectors
vectors = np.array([
    [5, 7, 9, 0],
    [2, 5, 0, 1],
    [0, 0, 0, 1],
    [7, 12, 9, 3]
])

# Using numpy to find the rank of the matrix composed of the given vectors
rank_of_matrix = np.linalg.matrix_rank(vectors)
rank_of_matrix
\end{lstlisting}
% 
% 
% 
% 
% 
\paragraph{\textbf{Analytics:} The four vectors $(5,7,9,0)$, $(2,5,0,1)$, $(0,0,0,1)$, and $(7,12,9,3)$ are actually linearly related, because the dimensions of the space they stretch are 3, not 2. This means that of the four vectors, at least one can be linearly represented by the other three. To find a set of linearly independent vectors, we need to remove at least one of the vectors so that the rank of the remaining set of vectors equals the number of vectors. In this example, since any three of these four vectors can form a basis of a stretched space, a linearly independent set of vectors can be obtained by removing any one of them.}
% 
% 
% 
% 
% 
\end{document}
% 