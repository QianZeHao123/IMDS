\section{Question 5}
\paragraph{Find a matrix that has eigenvalues 3; 4; 5 with corresponding
    eigenvectors: \\
    $\begin{pmatrix}
            1 \\ 0 \\ 1
        \end{pmatrix}$,
    $\begin{pmatrix}
            0 \\ 1 \\ 2
        \end{pmatrix}$,
    $\begin{pmatrix}
            0 \\ 1 \\ 10
        \end{pmatrix}$.
    Explain how you found this matrix.
}
\subsection{Analytics}
\paragraph{Assume tha our target matrix is $A$ and we have the eigenvalues and eigenvectors. So we can use $A=P \Lambda P^{-1}$ to calculate it}
\paragraph{Construct the matrix \(P\) using these eigenvectors as columns. The matrix \(P\) is given by:}
% 
% 
% 
% 
% 
% 
% 
\paragraph{
    $$
        P =
        \begin{pmatrix}
            \mathbf{v}_1 & \mathbf{v}_2 & \mathbf{v}_3
        \end{pmatrix} =
        \begin{pmatrix}
            1 & 0 & 0 \\
            0 & 1 & 1 \\
            1 & 2 & 0
        \end{pmatrix}
    $$
}
% 
% 
% 
% 
% 
% 
% 
$$$$
\begin{lstlisting}[style=pystyle]
import numpy as np

# Given eigenvalues
lambda_1 = 3
lambda_2 = 4
lambda_3 = 5

# Given eigenvectors
v1 = np.array([1, 0, 1])
v2 = np.array([0, 1, 2])
v3 = np.array([0, 1, 0])

# Construct the matrix A
P = np.column_stack((v1, v2, v3))
lambda_matrix = np.array(
    [[lambda_1, 0, 0], 
        [0, lambda_2, 0], 
        [0, 0, lambda_3]])

# Compute the matrix A with P * lambda * P^{-1}
A = P@lambda_matrix@np.linalg.inv(P)
print(A)
\end{lstlisting}
% 
% 
% 
% 
% 
\paragraph{\textbf{The matrix is:}
    $$
        \left[
            \begin{matrix}
                3.   & 0. & 0.   \\
                0.5  & 5. & -0.5 \\
                -1. & 0. & 4.
            \end{matrix}
            \right]
    $$}