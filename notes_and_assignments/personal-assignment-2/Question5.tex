\section{Question 5}
\paragraph{Find a matrix that has eigenvalues 3; 4; 5 with corresponding
    eigenvectors: \\
    $\begin{pmatrix}
            1 \\ 0 \\ 1
        \end{pmatrix}$,
    $\begin{pmatrix}
            0 \\ 1 \\ 2
        \end{pmatrix}$,
    $\begin{pmatrix}
            0 \\ 1 \\ 10
        \end{pmatrix}$.
    Explain how you found this matrix.
}
\subsection{Analytics}
\paragraph{Construct the matrix \(A\) using these eigenvectors as columns. The matrix \(A\) is given by:}
% 
% 
% 
% 
% 
% 
% 
\[ A = \begin{pmatrix} \mathbf{v}_1 & \mathbf{v}_2 & \mathbf{v}_3 \end{pmatrix} \]
% 
% 
% 
% 
% 
% 
% 
\begin{lstlisting}[style=pystyle]
import numpy as np

# Given eigenvalues
lambda_1 = 3
lambda_2 = 4
lambda_3 = 5

# Given eigenvectors
v1 = np.array([1, 0, 1])
v2 = np.array([0, 1, 2])
v3 = np.array([0, 1, 10])

# Construct the matrix A
A = np.column_stack((v1, v2, v3))

# Display the matrix A
print("Matrix A:")
print(A)

# Check if A has the correct eigenvalues and eigenvectors
for i in range(3):
    result = np.dot(A, A[:, i])
    print(f"Eigenvalue {i+1}:", result)
    
\end{lstlisting}
\paragraph{\textbf{Eigenvalue 1:} $\left[\begin{matrix}
                1 & 1 & 11
            \end{matrix}\right]$}
\paragraph{\textbf{Eigenvalue 2:} $\left[\begin{matrix}
                0 & 3 & 22
            \end{matrix}\right]$}
\paragraph{\textbf{Eigenvalue 3:} $\left[\begin{matrix}
                0 & 11 & 102
            \end{matrix}\right]$}
% 