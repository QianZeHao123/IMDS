\section{Question 4}
% \subsection{}
\paragraph{Consider a discrete dynamical system:
    $
        \begin{pmatrix}
            x_{i+1} \\
            y_{i+1} \\
            z_{i+1}
        \end{pmatrix} = A
        \begin{pmatrix}
            x_{i} \\
            y_{i} \\
            z_{i}
        \end{pmatrix}
    $.
    $$
        A=\begin{pmatrix}
            1.03552632  & 0.01842105 & -0.16447368 \\
            -0.00921053 & 1.10263158 & -0.00921053 \\
            -0.13815789 & 0.03947368 & 1.06184211
        \end{pmatrix}
    $$
}
\subsection{Caluclate the Next State}
\paragraph{If $(x_0,y_0,Z_0)=(100,100,100)$, caluclate $(x_1,y_1,z_1)$ and $(x_2,y_2,z_2)$}
\subsubsection{Analytics}
% 
$$$$
\begin{lstlisting}[style=pystyle]
import numpy as np

# Given matrix A
A = np.array([
    [1.03552632, 0.01842105, -0.16447368],
    [-0.00921053, 1.10263158, -0.00921053],
    [-0.13815789, 0.03947368, 1.06184211]
])

# Initial state (x0, y0, z0)
initial_state = np.array([100, 100, 100])

# Calculate the next states (x1, y1, z1) and (x2, y2, z2)
state_1 = np.dot(A, initial_state)
state_2 = np.dot(A, state_1)

# Print the results
print("Initial State:", initial_state)
print("Next State (x1, y1, z1):", state_1)
print("Next State (x2, y2, z2):", state_2)
\end{lstlisting}
% 
% 
% 
\subsubsection{Result}
\paragraph{Initial State: $[100\ 100\ 100]$ \\
    Next State (x1, y1, z1): $[ 88.947369\ 108.421052\ 96.31579 ]$ \\
    Next State (x2, y2, z2): $[ 78.26315889\ 117.84210399\ 94.26315877]$}
% 
% 
% 
% 
\subsection{Combination of Eigenvectors}
\paragraph{Given the initial state vector $(x_0, y_0, z_0)=(100, 100, 100)$, write this vector as a linear combination of eigenvectors of $A$.}
\subsubsection{Analytics}
% 
% 
% 
% 
% 
\paragraph{To express the initial state vector \((x_0, y_0, z_0) = (100, 100, 100)\) as a linear combination of the eigenvectors of matrix \(A\), we need to find the eigenvectors and eigenvalues of \(A\).}
% 
% 
% 
% 
% 
% 
\paragraph{let: $$|\lambda E - A| = 0 \Rightarrow \lambda_1,\lambda_2,\lambda_3$$}
\paragraph{Bring $\lambda_1,\lambda_2,\lambda_3$ back to matrix $(\lambda E - A)$ and get the Eigenvectors:$$
        \begin{pmatrix}
            v_{11} & v_{21} & v_{31} \\
            v_{12} & v_{22} & v_{32} \\
            v_{13} & v_{23} & v_{33}
        \end{pmatrix}
    $$}
% 
% 
% 
% 
% 
\paragraph{Let \(v_1, v_2, v_3\) be the eigenvectors of \(A\) corresponding to the eigenvalues \(\lambda_1, \lambda_2, \lambda_3\), respectively.}
% 
\paragraph{The expression for the initial state vector as a linear combination of eigenvectors is given by:}
% 
\paragraph{\[ \mathbf{x}_0 = c_1 \mathbf{v}_1 + c_2 \mathbf{v}_2 + c_3 \mathbf{v}_3 \]}
% 
% 
% 
% 
% 
% 
% 
\paragraph{$$
        \begin{pmatrix}
            x_0 \\ y_0 \\ z_0
        \end{pmatrix} =
        c_1 \begin{pmatrix}
            v_{11} \\ v_{12} \\ v_{13}
        \end{pmatrix} +
        c_2 \begin{pmatrix}
            v_{21} \\ v_{22} \\ v_{23}
        \end{pmatrix} +
        c_3 \begin{pmatrix}
            v_{31} \\ v_{32} \\ v_{33}
        \end{pmatrix}=
        \begin{pmatrix}
            v_{11} & v_{21} & v_{31} \\
            v_{12} & v_{22} & v_{32} \\
            v_{13} & v_{23} & v_{33}
        \end{pmatrix}
        \begin{pmatrix}
            c_1 \\ c_2 \\ c_3
        \end{pmatrix}
    $$}
% 
% 
% 
% 
% 
% 
% 
% 
% 
\paragraph{where \(c_1, c_2, c_3\) are the coefficients to be determined.}
% 
\paragraph{Then, calculate the eigenvectors and eigenvalues using NumPy:}
% 
$$$$
% 
\begin{lstlisting}[style=pystyle]
import numpy as np

# Given matrix A
A = np.array([
    [1.03552632, 0.01842105, -0.16447368],
    [-0.00921053, 1.10263158, -0.00921053],
    [-0.13815789, 0.03947368, 1.06184211]
])

# Calculate eigenvectors and eigenvalues
eigenvalues, eigenvectors = np.linalg.eig(A)

# Given initial state vector
x0 = np.array([100, 100, 100])

# Solve for coefficients c1, c2, c3
coefficients = np.linalg.solve(eigenvectors, x0)

# Print the results
print("Eigenvectors:")
print(eigenvectors)
print("\nEigenvalues:")
print(eigenvalues)
print("\nInitial state vector as a linear combination of eigenvectors:")
print(f"x0 = {coefficients[0]:.2f} * v1 + {coefficients[1]:.2f} * v2 + {coefficients[2]:.2f} * v3")
\end{lstlisting}
% 
\paragraph{This code calculates the eigenvectors and eigenvalues of matrix \(A\) and then solves for the coefficients \(c_1, c_2, c_3\) in the expression \(\mathbf{x}_0 = c_1 \mathbf{v}_1 + c_2 \mathbf{v}_2 + c_3 \mathbf{v}_3\).}
% 
\paragraph{\textbf{Eigenvalues:}
    $
        \left[ \begin{matrix}
                0.90000001 & 1.2 & 1.
            \end{matrix}\right]
    $}
\paragraph{\textbf{Eigenvectors:}
    $
        \left[
            \begin{matrix}
                7.66651880e-01 & 7.07106781e-01  & -2.74721110e-01 \\
                6.38876847e-02 & 4.18086088e-16  & -9.61523953e-01 \\
                6.38876559e-01 & -7.07106781e-01 & 9.94056696e-10
            \end{matrix}
            \right]
    $}
\paragraph{\textbf{Initial state vector as a linear combination of eigenvectors:}$$ x_0 = 123.57 v_1 -29.77 v_2  -95.79 v_3 $$}
% 
% 
% 
% 
% 
% 
% 
% 
\subsection{Finding Initial State for Decreasing Norm}
\paragraph{Find an initial state vector $(x_0,y_0,z_0)$ of norm 10 such that the norm of $(x_i,y_i,z_i)$ get smaller and smaller as $i \rightarrow \infty$.}
% 
% 
% 
% 
\paragraph{The general form of the solution for a discrete dynamical system $\mathbf{v}_{i+1} = A \mathbf{v}_i $  is
    $$ \mathbf{v}_i = c_1 \lambda_1^i \mathbf{v}_1 + c_2 \lambda_2^i \mathbf{v}_2 + c_3 \lambda_3^i \mathbf{v}_3 $$}
% 
% 
%
% 
% 
\paragraph{where $ \lambda_1, \lambda_2, \lambda_3 $ are the eigenvalues of \(A\) and \(\mathbf{v}_1, \mathbf{v}_2, \mathbf{v}_3\) are the corresponding eigenvectors.}
% 
% 
% 
%
\paragraph{To achieve a decreasing norm as \(i \rightarrow \infty\), we need to ensure that the eigenvalues have absolute values less than 1. If \(|\lambda_i| < 1\), then \(\lim_{i \to \infty} \lambda_i^i = 0\), and the norm of \(\mathbf{v}_i\) will decrease.}
% 
% 
% 
%
$$$$
% 
% 
% 
% 
\begin{lstlisting}[style=pystyle]
import numpy as np

# Given matrix A
A = np.array([
    [1.03552632, 0.01842105, -0.16447368],
    [-0.00921053, 1.10263158, -0.00921053],
    [-0.13815789, 0.03947368, 1.06184211]
])

# Calculate eigenvectors and eigenvalues
eigenvalues, eigenvectors = np.linalg.eig(A)

# Find eigenvector corresponding to the eigenvalue with the maximum absolute value
min_abs_eigenvalue_index = np.argmin(np.abs(eigenvalues))
initial_eigenvector = eigenvectors[:, min_abs_eigenvalue_index]

# Normalize the eigenvector to have a norm of 10
initial_state = 10 * (initial_eigenvector /
                        np.linalg.norm(initial_eigenvector))

# Print the results
print("Initial state vector with norm 10:", initial_state)

\end{lstlisting}
% 
% 
% 
%
% 
% 
% 
%
\paragraph{\textbf{Initial state vector with norm 10:}
$$
\left[
    \begin{matrix}
        7.6665188 & 0.63887685 & 6.38876559
    \end{matrix}
    \right]
$$}