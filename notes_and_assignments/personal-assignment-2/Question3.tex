\section{Question 3}
\subsection{Find Value for x and y}
\paragraph{Consider the linear system of equations
    $$ \begin{pmatrix}
            6 & -4 \\
            8 & -6 \\
            2 & 0
        \end{pmatrix} \begin{pmatrix}
            x \\
            y
        \end{pmatrix} = \begin{pmatrix}
            1 \\
            1 \\
            1
        \end{pmatrix} $$
    Either find a solution or explain why it does not have any solutions.
}
\subsection{Analyics}
\paragraph{To determine whether the linear system has a solution, we can check the consistency of the system using the augmented matrix. The augmented matrix for the given system is:
    $$ \begin{pmatrix}
            6 & -4 & 1 \\
            8 & -6 & 1 \\
            2 & 0  & 1
        \end{pmatrix} $$}
% \subsection{Gradient Estimation at $(0.5, 0.3)$ Using Linear Interpolation}
% \paragraph{In this question I use the \textbf{finite difference method}. It is a commonly used numerical computational technique for estimating derivatives or partial derivatives of a function, especially when dealing with discrete data points.}
% % 
% % 
% \paragraph{The finite difference method is typically applied in the following scenarios:}
% % 
% % 
% \begin{itemize}
%     \item A set of discrete data points but do not possess an analytical expression for the function.
%     \item Compute the derivative of a function at a specific point, but lack an analytical expression for the derivative.
% \end{itemize}
% % 
% \paragraph{I apply Central Difference Method (Here, h is a small step size):}
% % 
% % 
% % 
% \begin{equation}
%     \nabla g(x,y)=(\frac{\partial g}{\partial x},\frac{\partial g}{\partial y})
% \end{equation}
% % 
% % 
% \begin{equation}
%     \frac{\partial g}{\partial x}=\frac{g(x+h,y)-g(x-h,y)}{2 h}
% \end{equation}
% % 
% % 
% \begin{equation}
%     \frac{\partial g}{\partial y}=\frac{g(x,y+h)-g(x,y-h)}{2 h}
% \end{equation}
% % 
% % 
% % 
% % 
% % 
% \begin{align*}
%     \frac{\partial g_{x=0.5, y=0.3}}{\partial x_{x=0.5}} & = \frac{g(0.6,0.3)-g(0.4,0.3)}{2 \times 0.1} \\
%                                                          & = \frac{0.003-(-0.337)}{0.2}                 \\
%                                                          & =1.7
% \end{align*}
% % 
% % 
% % 
% \begin{align*}
%     \frac{\partial g_{x=0.5, y=0.3}}{\partial y_{y=0.3}} & = \frac{g(0.5,0.4)-g(0.5,0.2)}{2 \times 0.1} \\
%                                                          & = \frac{-0.234-(-0.038)}{0.2}                \\
%                                                          & = -0.98
% \end{align*}
% % 
% % 
% % 
% \paragraph{So, $\nabla g(x_{x=0.5},y_{y=0.3})=(1.7,-0.98)$.}
% % 
% % 
% % 
% % 
% \subsection{Estimating Directional Derivative of g at $(0.5, 0.3)$ along the Vector $u=(1,-4)$}
% % 
% % 
% % 
% % 
% \paragraph{To estimate the directional derivative of \(g\) at the point \((x, y) = (0.5, 0.3)\) along the vector \(\mathbf{u} = (1, -4)\), the directional derivative of \(g\) along the vector \(\mathbf{u} = (a, b)\) can be calculated using the gradient \(\nabla g\) as follows ((a,b) is the unitized $\vec{u}$.):}

% \begin{equation}
%     D_{\mathbf{u}} g = \nabla g \cdot \mathbf{u} = \frac{\partial g}{\partial x} \cdot a + \frac{\partial g}{\partial y} \cdot b
% \end{equation}

% \paragraph{\(\mathbf{u} = (1, -4)\) and I have estimated the gradient components as \(\frac{\partial g}{\partial x}\) and \(\frac{\partial g}{\partial y}\).}
% % 
% % 
% % 
% %     
% $$ \vec{u}=(\frac{u_x}{|u|},\frac{u_y}{|u|})=(\frac{1}{\sqrt{17}},\frac{-4}{\sqrt{17}}) $$
% % 
% % 
% % 
% $$ D_{\mathbf{u}} g = \left(\frac{\partial g}{\partial x}\right) \cdot \frac{1}{\sqrt{17}} + \left(\frac{\partial g}{\partial y}\right) \cdot (\frac{-4}{\sqrt{17}}) $$

% \paragraph{Use the values for \(\frac{\partial g}{\partial x}\) and \(\frac{\partial g}{\partial y}\) to calculate \(D_{\mathbf{u}} g\).}
% % 
% % 
% % 
% $$ D_u g = 1.7 \times \frac{1}{\sqrt{17}} - 0.98 \times (\frac{-4}{\sqrt{17}}) = 1.3507 $$
% % 
% % 
% % 
% % 
% % 
% % 
% % 
% % 