% !TEX TS-program = pdflatex
% !TEX encoding = UTF-8 Unicode
% 
% This is a simple template for a LaTeX document using the "article" class.
% See "book", "report", "letter" for other types of document.
% 
\documentclass[11pt]{article} % use larger type; default would be 10pt
% 
% 
% 
%%% The "real" document content comes below...
\usepackage[utf8]{inputenc} % set input encoding (not needed with XeLaTeX)

%%% Examples of Article customizations
% These packages are optional, depending whether you want the features they provide.
% See the LaTeX Companion or other references for full information.

%%% PAGE DIMENSIONS
\usepackage{geometry} % to change the page dimensions
\geometry{a4paper} % or letterpaper (US) or a5paper or....
% \geometry{margin=2in} % for example, change the margins to 2 inches all round
% \geometry{landscape} % set up the page for landscape
%   read geometry.pdf for detailed page layout information

\usepackage{graphicx} % support the \includegraphics command and options

% \usepackage[parfill]{parskip} % Activate to begin paragraphs with an empty line rather than an indent

%%% PACKAGES
\usepackage{booktabs} % for much better looking tables
\usepackage{array} % for better arrays (eg matrices) in maths
\usepackage{paralist} % very flexible & customisable lists (eg. enumerate/itemize, etc.)
\usepackage{verbatim} % adds environment for commenting out blocks of text & for better verbatim
\usepackage{subfig} % make it possible to include more than one captioned figure/table in a single float
% These packages are all incorporated in the memoir class to one degree or another...

%%% HEADERS & FOOTERS
\usepackage{fancyhdr} % This should be set AFTER setting up the page geometry
\pagestyle{fancy} % options: empty , plain , fancy
\renewcommand{\headrulewidth}{0pt} % customise the layout...
\lhead{}\chead{}\rhead{}
\lfoot{}\cfoot{\thepage}\rfoot{}

%%% SECTION TITLE APPEARANCE
\usepackage{sectsty}
\allsectionsfont{\sffamily\mdseries\upshape} % (See the fntguide.pdf for font help)
% (This matches ConTeXt defaults)

%%% ToC (table of contents) APPEARANCE
\usepackage[nottoc,notlof,notlot]{tocbibind} % Put the bibliography in the ToC
\usepackage[titles,subfigure]{tocloft} % Alter the style of the Table of Contents
\renewcommand{\cftsecfont}{\rmfamily\mdseries\upshape}
\renewcommand{\cftsecpagefont}{\rmfamily\mdseries\upshape} % No bold!

%%% END Article customizations
\usepackage{xcolor}
\usepackage{tcolorbox}
\usepackage{lipsum}  % 示例文本
\usepackage{mdframed}

\usepackage{tikz}

% 
\title{Introduction to Math for DS Group Task 1}
\author{IMDS Group 24 \\ Zehao Qian, Mohammad Jamshaid Iqbal, Chloe Mendez}
\begin{document}
\maketitle
% 
% 
% \input{example}
\section{Question 1}
\paragraph{Consider the line L in $R^3$ passing through the origin and the point P=(2,2,1)? Which of the following points is closest to L, and which is farthest from L?}
\begin{itemize}
    \item A: (8, 9, 0.9)
    \item B: (4, 4, 2.1)
    \item C: (0.9, 0.99, 0.49)
    \item D: (-2, -2, 1)
    \item E: (0, 2.1, 0)
\end{itemize}

\subsection{Analytics:}
\paragraph{In three-dimensional space, we can calculate the distance from a point to a line using the following formula:}
\paragraph{Let's say you have a point P with coordinates $(x_0, y_0, z_0)$, and a line defined by two points A$(x_1, y_1, z_1)$ and B$(x_2, y_2, z_2)$. The formula to calculate the distance (d) from point P to the line AB is as follows:}
% 
% 
$$ d = \frac{|(\mathbf{P} - \mathbf{A}) \cdot \mathbf{n}|}{|\mathbf{AB}|} $$
% 
% 
\paragraph{Here's an explanation of the formula components:}
% 
% 
\begin{itemize}
    \item \(\mathbf{P} - \mathbf{A}\) represents the vector from point A to point P.
    \item \(\mathbf{AB}\) represents the vector along the line from point A to point B.
    \item \(\cdot\) denotes the dot product between vectors.
    \item \(\mathbf{n}\) is the unit vector along the line AB, which is given by \(\frac{\mathbf{AB}}{|\mathbf{AB}|}\).
\end{itemize}
% 
% 



\tikzset{every picture/.style={line width=0.75pt}} %set default line width to 0.75pt        

\begin{tikzpicture}[x=0.75pt,y=0.75pt,yscale=-1,xscale=1]
%uncomment if require: \path (0,263); %set diagram left start at 0, and has height of 263

%Shape: Axis 2D [id:dp44317292301248945] 
\draw  (166.5,203.07) -- (468.9,203.07)(196.74,24.92) -- (196.74,222.87) (461.9,198.07) -- (468.9,203.07) -- (461.9,208.07) (191.74,31.92) -- (196.74,24.92) -- (201.74,31.92)  ;
%Straight Lines [id:da2979242010824248] 
\draw    (197.86,202.72) -- (363.74,67.68) ;
%Shape: Ellipse [id:dp023999541748674025] 
\draw  [fill={rgb, 255:red, 0; green, 0; blue, 0 }  ,fill opacity=1 ] (310.87,219.7) .. controls (309.68,219.31) and (309.04,218.07) .. (309.45,216.93) .. controls (309.86,215.8) and (311.16,215.19) .. (312.35,215.58) .. controls (313.54,215.97) and (314.17,217.21) .. (313.76,218.35) .. controls (313.35,219.48) and (312.06,220.09) .. (310.87,219.7) -- cycle ;
%Straight Lines [id:da4396600304527325] 
\draw    (274.28,123.97) -- (252.25,139.71) ;
\draw [shift={(250.62,140.87)}, rotate = 324.46] [color={rgb, 255:red, 0; green, 0; blue, 0 }  ][line width=0.75]    (10.93,-3.29) .. controls (6.95,-1.4) and (3.31,-0.3) .. (0,0) .. controls (3.31,0.3) and (6.95,1.4) .. (10.93,3.29)   ;
%Straight Lines [id:da11083812462923981] 
\draw  [dash pattern={on 0.84pt off 2.51pt}]  (197.86,202.72) -- (309.45,216.93) ;
%Straight Lines [id:da3033500048537672] 
\draw [color={rgb, 255:red, 65; green, 117; blue, 5 }  ,draw opacity=1 ] [dash pattern={on 0.84pt off 2.51pt}]  (265.1,149.52) -- (313.76,218.35) ;
%Shape: Brace [id:dp9223930624399752] 
\draw   (312.19,216.89) .. controls (316.06,214.28) and (316.69,211.04) .. (314.08,207.17) -- (300.55,187.1) .. controls (296.82,181.57) and (296.9,177.51) .. (300.77,174.9) .. controls (296.9,177.51) and (293.1,176.05) .. (289.37,170.52)(291.05,173) -- (275.85,150.45) .. controls (273.24,146.58) and (270,145.95) .. (266.13,148.56) ;
%Straight Lines [id:da5740957717017106] 
\draw    (256.74,153.91) -- (261.33,161.91) -- (270.72,156.55) ;
%Straight Lines [id:da9385906535715354] 
\draw    (196.74,203.07) -- (154.47,242.1) ;
\draw [shift={(153,243.45)}, rotate = 317.29] [color={rgb, 255:red, 0; green, 0; blue, 0 }  ][line width=0.75]    (10.93,-3.29) .. controls (6.95,-1.4) and (3.31,-0.3) .. (0,0) .. controls (3.31,0.3) and (6.95,1.4) .. (10.93,3.29)   ;

% Text Node
\draw (175.39,180.91) node [anchor=north west][inner sep=0.75pt]   [align=left] {O};
% Text Node
\draw (358.93,55.68) node [anchor=north west][inner sep=0.75pt]  [rotate=-320.43] [align=left] {P};
% Text Node
\draw (314.8,218.52) node [anchor=north west][inner sep=0.75pt]  [rotate=-359.39] [align=left] {X};
% Text Node
\draw (242.82,116.02) node [anchor=north west][inner sep=0.75pt]  [font=\footnotesize,rotate=-327.31]  {$\vec{n}$};
% Text Node
\draw (297.75,160.53) node [anchor=north west][inner sep=0.75pt]  [rotate=-321.21]  {$d$};
% Text Node
\draw (242.17,211.95) node [anchor=north west][inner sep=0.75pt]  [font=\footnotesize,rotate=-8.77]  {$\overrightarrow{OX}$};
% Text Node
\draw (452.87,174.47) node [anchor=north west][inner sep=0.75pt]   [align=left] {X};
% Text Node
\draw (199.49,15.43) node [anchor=north west][inner sep=0.75pt]   [align=left] {Y};
% Text Node
\draw (135.87,227.47) node [anchor=north west][inner sep=0.75pt]   [align=left] {Z};


\end{tikzpicture}
% 
% 
% 
% 
% 
% 
% 
% 
% 
% 
% 
% 
% 
% 
% 
% 
% 
% 
% 
% 
% 
\section{Question 2}
\paragraph{Consider the function: }
$$ f(x,y)=e^{\frac{-(x^2+y^2)}{30}} \sin \frac{x^2+y^2}{5} $$
\paragraph{If we perform a single iteration of gradient descent starting at the point (5,-6), will we move further from the origin, or closer to it?}
% 
\subsection{Analytics:}
% 
% 
% 
\paragraph{The gradient of \(f(x, y)\):}

$$
\nabla f(x, y) = \left(\frac{\partial f}{\partial x}, \frac{\partial f}{\partial y}\right)
$$

\paragraph{1. Calculate the partial derivative of \(f(x, y)\) with respect to \(x\):}
% 
$$ \frac{\partial f}{\partial x} = e^{\frac{-(x^2+y^2)}{30}}\left(-\frac{2x}{30}\right) \sin \frac{x^2+y^2}{5} + e^{\frac{-(x^2+y^2)}{30}} \cos \frac{x^2+y^2}{5} \left(\frac{2x}{5}\right) $$
% 
\paragraph{2. Calculate the partial derivative of \(f(x, y)\) with respect to \(y\):}
$$
\frac{\partial f}{\partial y} = e^{\frac{-(x^2+y^2)}{30}}\left(-\frac{2y}{30}\right) \sin \frac{x^2+y^2}{5} + e^{\frac{-(x^2+y^2)}{30}} \cos \frac{x^2+y^2}{5} \left(\frac{2y}{5}\right)
$$
% 
\paragraph{The gradient at the point (5, -6):}

$$
\nabla f(5, -6) = \left(\frac{\partial f}{\partial x}(5, -6), \frac{\partial f}{\partial y}(5, -6)\right)
$$

\paragraph{Next, we calculate the negative gradient:}
% 
$$
-\nabla f(5, -6) = -\left(\frac{\partial f}{\partial x}(5, -6), \frac{\partial f}{\partial y}(5, -6)\right)
$$
% 
\paragraph{This negative gradient vector points in the direction of steepest decrease. If this vector points away from the origin, then a single iteration of gradient descent will move further from the origin. If it points towards the origin, it will move closer to the origin. You can analyze the components of the negative gradient to determine the direction.}
% 
% 
% 
% 
% 
\end{document}
% 