% !TEX TS-program = pdflatex
% !TEX encoding = UTF-8 Unicode
% 
% This is a simple template for a LaTeX document using the "article" class.
% See "book", "report", "letter" for other types of document.
% 
\documentclass[11pt]{article} % use larger type; default would be 10pt
% 
% 
% 
%%% The "real" document content comes below...
\usepackage[utf8]{inputenc} % set input encoding (not needed with XeLaTeX)

%%% Examples of Article customizations
% These packages are optional, depending whether you want the features they provide.
% See the LaTeX Companion or other references for full information.

%%% PAGE DIMENSIONS
\usepackage{geometry} % to change the page dimensions
\geometry{a4paper} % or letterpaper (US) or a5paper or....
% \geometry{margin=2in} % for example, change the margins to 2 inches all round
% \geometry{landscape} % set up the page for landscape
%   read geometry.pdf for detailed page layout information

\usepackage{graphicx} % support the \includegraphics command and options

% \usepackage[parfill]{parskip} % Activate to begin paragraphs with an empty line rather than an indent

%%% PACKAGES
\usepackage{booktabs} % for much better looking tables
\usepackage{array} % for better arrays (eg matrices) in maths
\usepackage{paralist} % very flexible & customisable lists (eg. enumerate/itemize, etc.)
\usepackage{verbatim} % adds environment for commenting out blocks of text & for better verbatim
\usepackage{subfig} % make it possible to include more than one captioned figure/table in a single float
% These packages are all incorporated in the memoir class to one degree or another...

%%% HEADERS & FOOTERS
\usepackage{fancyhdr} % This should be set AFTER setting up the page geometry
\pagestyle{fancy} % options: empty , plain , fancy
\renewcommand{\headrulewidth}{0pt} % customise the layout...
\lhead{}\chead{}\rhead{}
\lfoot{}\cfoot{\thepage}\rfoot{}

%%% SECTION TITLE APPEARANCE
\usepackage{sectsty}
\allsectionsfont{\sffamily\mdseries\upshape} % (See the fntguide.pdf for font help)
% (This matches ConTeXt defaults)

%%% ToC (table of contents) APPEARANCE
\usepackage[nottoc,notlof,notlot]{tocbibind} % Put the bibliography in the ToC
\usepackage[titles,subfigure]{tocloft} % Alter the style of the Table of Contents
\renewcommand{\cftsecfont}{\rmfamily\mdseries\upshape}
\renewcommand{\cftsecpagefont}{\rmfamily\mdseries\upshape} % No bold!

%%% END Article customizations
\usepackage{xcolor}
\usepackage{tcolorbox}
\usepackage{lipsum}  % 示例文本
\usepackage{mdframed}

\usepackage{tikz}

% 
\title{Introduction to Math for DS Group Task 3}
\author{IMDS Group 24 \\ Zehao Qian, Mohammad Jamshaid Iqbal, Chloe Mendez}
\begin{document}
\maketitle
% 
% 
% \input{example}
\section{Question 1}

\paragraph{Are the vectors $(1,3,4)$, $(4,-1,5)$ and $(5,2,9)$ linearly independent?}


\subsection{AI Bot's Answer}
\begin{mdframed}[
        backgroundcolor=white,  % 背景颜色
        linecolor=black,        % 边框颜色
        leftmargin=5pt,         % 左边距
        rightmargin=5pt,        % 右边距
        linewidth=2pt           % 边框的宽度
    ]
    \paragraph{\textbf{ChatGPT}: To determine if the vectors \( \begin{bmatrix} 1 \\ 3 \\ 4 \end{bmatrix} \), \( \begin{bmatrix} 4 \\ -1 \\ 5 \end{bmatrix} \), and \( \begin{bmatrix} 5 \\ 2 \\ 9 \end{bmatrix} \) are linearly independent, you can create a matrix using these vectors and perform row operations to see if the matrix reduces to the row-echelon form with non-zero rows.}
    \paragraph{Let \( A \) be the matrix with the given vectors as its columns:}
    % 
    % 
    \[ A = \begin{bmatrix} 1 & 4 & 5 \\ 3 & -1 & 2 \\ 4 & 5 & 9 \end{bmatrix} \]
    % 
    % 
    \paragraph{Now, perform row operations to get the row-echelon form:}
    \[ \begin{bmatrix} 1 & 4 & 5 \\ 3 & -1 & 2 \\ 4 & 5 & 9 \end{bmatrix} \]
    % 
    \[ \overset{R_2 = R_2 - 3R_1}{\longrightarrow} \begin{bmatrix} 1 & 4 & 5 \\ 0 & -13 & -13 \\ 4 & 5 & 9 \end{bmatrix} \]
    % 
    \[ \overset{R_3 = R_3 - 4R_1}{\longrightarrow} \begin{bmatrix} 1 & 4 & 5 \\ 0 & -13 & -13 \\ 0 & -11 & -11 \end{bmatrix} \]
    % 
    \[ \overset{R_3 = R_3 - \frac{11}{13}R_2}{\longrightarrow} \begin{bmatrix} 1 & 4 & 5 \\ 0 & -13 & -13 \\ 0 & 0 & 0 \end{bmatrix} \]
    % 
    % 
    \paragraph{Now, we have the row-echelon form. Since there is a row of zeros, the system is linearly dependent. Specifically, the third vector can be expressed as a linear combination of the first two vectors.}
    % 
    % 
    \paragraph{Therefore, the vectors \( \begin{bmatrix} 1 \\ 3 \\ 4 \end{bmatrix} \), \( \begin{bmatrix} 4 \\ -1 \\ 5 \end{bmatrix} \), and \( \begin{bmatrix} 5 \\ 2 \\ 9 \end{bmatrix} \) are linearly dependent.}
\end{mdframed}
% 
% 
% 
% 
% 
% 
% 
% 
% 
% 
% 
% 
% 
% 
% 
% 
% 
% 
% 
% 
% 
% 
% 
% 
% 
\section{Question 2}
% 
\paragraph{Is the vector $(1,0,-1)$ in the plane spanned by the vectors $(4,11,7)$ and $(3,11,8)$?}
% 
% 
\subsection{AI Bot's Answer}
\begin{mdframed}[
        backgroundcolor=white,  % 背景颜色
        linecolor=black,        % 边框颜色
        leftmargin=5pt,         % 左边距
        rightmargin=5pt,        % 右边距
        linewidth=2pt           % 边框的宽度
    ]
    \paragraph{\textbf{ChatGPT}: To determine whether the vector \( (1, 0, -1) \) is in the plane spanned by the vectors \( (4, 11, 7) \) and \( (3, 11, 8) \), we can check if \( (1, 0, -1) \) can be expressed as a linear combination of the other two vectors.}
    % 
    % 
    \paragraph{Let \( A \) be the matrix formed by the vectors \( (4, 11, 7) \) and \( (3, 11, 8) \), and \( \mathbf{v} \) be the vector \( (1, 0, -1) \):}
    % 
    % 
    \[ A = \begin{bmatrix} 4 & 3 \\ 11 & 11 \\ 7 & 8 \end{bmatrix} \]
    % 
    % 
    \paragraph{We want to check if there exist scalars \( x \) and \( y \) such that \( A \begin{bmatrix} x \\ y \end{bmatrix} = \begin{bmatrix} 1 \\ 0 \\ -1 \end{bmatrix} \).}
    % 
    % 
    \paragraph{Solving the system of equations:}
    % 
    % 
    \[ \begin{bmatrix} 4 & 3 \\ 11 & 11 \\ 7 & 8 \end{bmatrix} \begin{bmatrix} x \\ y \end{bmatrix} = \begin{bmatrix} 1 \\ 0 \\ -1 \end{bmatrix} \]
    % 
    % 
    \paragraph{This system is inconsistent (no solution) because the augmented matrix \([A | \mathbf{v}]\) does not reduce to the identity matrix. Therefore, \( (1, 0, -1) \) is not in the plane spanned by \( (4, 11, 7) \) and \( (3, 11, 8) \).}
\end{mdframed}
% 
% 
% 
% 
% 
% 
% % 
% % 
% \begin{lstlisting}[style=pystyle]
% import numpy as np

% # Define the matrices A and B
% A = np.array([
%     [1, 0, 4, 1],
%     [0, 2, 0, 2],
%     [6, 0, 3, 11]
% ])

% B = np.array([
%     [7, -1, 2],
%     [1, 1, 0],
%     [2, 0, 1]
% ])

% # Calculate the product of A and B
% AB_product = np.dot(A, B)
% AB_product    
% \end{lstlisting}
% % 
% % 
% % 
% % 
% \paragraph{\textcolor{red}{Return ValueError: shapes (3,4) and (3,3) not aligned: 4 (dim 1) != 3 (dim 0)}}
% % 
% % 
% \paragraph{\textbf{Analytics:} The product of two matrices AB is undefined if the number of columns in the first matrix A does not match the number of rows in the second matrix B. In this case, matrix A has 4 columns, while matrix B has 3 rows, so their product cannot be computed. Matrix multiplication requires that the number of columns in the first matrix be equal to the number of rows in the second matrix. If there's a third matrix C that should be involved to make the multiplication possible, please provide it, otherwise matrix A and B as given cannot be multiplied.}
% % 
% % 
% % 
% % 
% % 
% % 
% % 
% % 
% % 
% % 
% % 
% % 
% % 
% % 
% \section{Question 2}
% \paragraph{What is the dimension of the span of the vectors $(5,7,9,0)$, $(2,5,0,1)$, $(0,0,0,1)$ and $(7,12,9,3)$?}
% % 
% % 
% % 
% % 
% % 
% % 
% \subsection{AI Bot's Answer}
% \begin{mdframed}[
%         backgroundcolor=white,  % 背景颜色
%         linecolor=black,        % 边框颜色
%         leftmargin=5pt,         % 左边距
%         rightmargin=5pt,        % 右边距
%         linewidth=2pt           % 边框的宽度
%     ]
%     \paragraph{\textbf{ChatGPT}: To find the dimension of the span of the given vectors, we can create a matrix using these vectors as columns and then find the rank of the matrix.}
%     % 
%     \paragraph{The given vectors are:}
%     % 
%     \[
%         \begin{bmatrix}
%             5 \\
%             7 \\
%             9 \\
%             0
%         \end{bmatrix},
%         \begin{bmatrix}
%             2 \\
%             5 \\
%             0 \\
%             1
%         \end{bmatrix},
%         \begin{bmatrix}
%             0 \\
%             0 \\
%             0 \\
%             1
%         \end{bmatrix},
%         \begin{bmatrix}
%             7  \\
%             12 \\
%             9  \\
%             3
%         \end{bmatrix}
%     \]
%     % 
%     \paragraph{Let's create a matrix \(M\) using these vectors:}

%     \[ M = \begin{bmatrix}
%             5 & 2 & 0 & 7  \\
%             7 & 5 & 0 & 12 \\
%             9 & 0 & 0 & 9  \\
%             0 & 1 & 1 & 3
%         \end{bmatrix} \]
%     % 
%     \paragraph{Now, we will perform row operations to bring the matrix to its echelon form or reduced row-echelon form and find the rank.}
%     % 
%     \paragraph{\textcolor{red}{To determine how many linearly independent vectors are among these four vectors, arrange them into a matrix and perform row operations to transform it into row-echelon form (REF) or reduced row-echelon form (RREF). ChatGPT choose the correct answer.}}
%     % 
%     \[ RREF(M) = \begin{bmatrix}
%             1 & 0 & 0 & 1 \\
%             0 & 1 & 1 & 2 \\
%             0 & 0 & 0 & 0 \\
%             0 & 0 & 0 & 0
%         \end{bmatrix} \]

%     \paragraph{The rank of this matrix (and hence the vectors) is 2. Therefore, the dimension of the span of the given vectors is 2.}
%     % 
%     \paragraph{\textcolor{red}{However, ChatGPT made an error in calculating RREF(M) during the matrix transformation, consequently leading to an incorrect rank for the matrix.}}
%     % 
%     $$$$

% \end{mdframed}
% % 
% % 
% % 
% % 
% % 
% % 
% % 
% % 
% % 
% % 
% % 
% % 
% \subsection{Analytics}
% % 
% % 
% % 
% \paragraph{We used the "np.linalg.matrix\_rank" function in NumPy to calculate the rank of matrix \(M\).}
% % 
% $$$$
% % 
% % 
% \begin{lstlisting}[style=pystyle]
% import numpy as np

% # Define the vectors
% vectors = np.array([
%     [5, 7, 9, 0],
%     [2, 5, 0, 1],
%     [0, 0, 0, 1],
%     [7, 12, 9, 3]
% ])

% # Using numpy to find the rank of the matrix composed of the given vectors
% rank_of_matrix = np.linalg.matrix_rank(vectors)
% rank_of_matrix
% \end{lstlisting}
% % 
% % 
% % 
% % 
% % 
% \paragraph{\textbf{Analytics:} The four vectors $(5,7,9,0)$, $(2,5,0,1)$, $(0,0,0,1)$, and $(7,12,9,3)$ are actually linearly related, because the dimensions of the space they stretch are 3, not 2. This means that of the four vectors, at least one can be linearly represented by the other three. To find a set of linearly independent vectors, we need to remove at least one of the vectors so that the rank of the remaining set of vectors equals the number of vectors. In this example, since any three of these four vectors can form a basis of a stretched space, a linearly independent set of vectors can be obtained by removing any one of them.}
% 
% 
% 
% 
% 
\end{document}
% 